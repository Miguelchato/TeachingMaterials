\documentclass[11pt]{article}
\usepackage[a4paper,top=3cm,bottom=3cm,left=2cm,right=2cm]{geometry}
%\usepackage[authoryear,round]{natbib}
\usepackage{hyperref}
\usepackage[pdftex]{color,graphicx,epsfig}
\DeclareGraphicsRule{.pdftex}{pdf}{.pdftex}{}
\usepackage{amssymb,amsmath}
\usepackage{enumerate}
\usepackage[spanish]{babel}
\usepackage[ansinew]{inputenc}



\begin{document}

%\setkeys{Gin}{width=0.99\textwidth}

\title{\bf Ejercicios Datos Longitudinales (II)}
\date{}

%\VignetteIndexEntry{RNAseq}

\maketitle


%%%%%%%%%%%%%%%%%%%%%%%%%%%%%%%%%%%%%%%%%%%%%%%%%%%%%%%%%%%%%%%%%%%


%%%%%%%%%%%%%%%%%%%%%%%%%%%%%%%%%%%%%%%%%%%%%%%%%%%%%%%%%%%%%%%%%%%
\noindent \textbf{PROBLEMA 1.}
Para estudiar las diferencias entre dos procedimientos diferentes
de recuperaci�n de pacientes de un infarto, se consideraron dos grupos experimentales en
sendos hospitales, de 8 pacientes cada uno. La variable respuesta es el �ndice de Bartel, que
var�a entre 0 y 100, y que constituye una medida de la habilidad funcional con la que se
valoran diferentes capacidades, de forma que valores m�s altos se corresponden con una
mejor situaci�n del paciente. De cada uno de los 16 pacientes se dispone de su respuesta cada
semana a lo largo de 5 semanas consecutivas. Los datos se pueden encontrar en el archivo
{\tt recuperainfarto.txt}

\medskip

\begin{enumerate}
\item Utiliza un modelo GEE para ver si existen diferencias estad�sticamente significativas entre los dos procedimientos empleados. Explicita qu� matriz de correlaci�n has usado par contestar a esta pregunta

\item Contesta a la misma pregunta utilizando un modelo lineal mixto �se obtiene la misma conclusi�n?
\end{enumerate}

\bigskip
\bigskip

\noindent \textbf{PROBLEMA 2.}

En un estudio sobre la agudeza visual se dispone de la respuesta de siete
individuos. La respuesta en cada ojo es el retraso en milisegundos entre la emisi�n de una luz
y la respuesta en a la misma por el cortex. Cada ojo se somete a cuatro mediciones correspondientes a cuatro instantes consecutivos. Se tienen mediciones en el ojo izquierdo y derecho. Los datos se pueden encontrar en el archivo {\tt agudezavisual.txt}

\begin{enumerate}
\item Crea un objeto {\tt groupeData} teniendo en cuenta que dispones de datos anidados (ojo dentro de medida en cada instante de tiempo)  NOTA: Usa `/' para indicar que una observaci�n est� anidada dentro de la otra 
\item Crea un gr�fico que ilustre el perfil de cada individuo y para cada ojo    
\item �Existe un efecto temporal en la respuesta ?
\end{enumerate}





\end{document}

%%%%%%%%%%%%%%%%%%%%%%%%%%%%%%%%%%%%%%%%%%%%%%%%%%%%%%%%%%%%%%%%%%%%%%%%%%%%%%%%%%%%%%%%%%%%%%%%%%%%%%%%%%%%%%%%%%% 