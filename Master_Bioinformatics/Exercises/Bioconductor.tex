\documentclass[11pt]{article}
\usepackage[a4paper,top=3cm,bottom=3cm,left=2cm,right=2cm]{geometry}
%\usepackage[authoryear,round]{natbib}
\usepackage{hyperref}
\usepackage[pdftex]{color,graphicx,epsfig}
\DeclareGraphicsRule{.pdftex}{pdf}{.pdftex}{}
\usepackage{amssymb,amsmath}
\usepackage{enumerate}
\usepackage[spanish]{babel}
\usepackage[ansinew]{inputenc}



\begin{document}


\title{\bf Bioconductor}
\date{}


\maketitle



\begin{enumerate}
\item Retrieve a matrix containing the gene counts located in the chromosome 7 in the region 5Mb-12Mb of airway data using  two R code: one using `[` and another using {\tt GRanges} functionalities.  

\item Find the gene symbol, chromosome position and GO pathway
ID for `1003\_s\_at` (Affymetrix probe annotated in {\tt hgu95av.db})

\item Use what you have learned about {\tt biomaRt} to find the gene symbol and name for the entrez gene IDs 1, 10 and 100.

\item Add the annotation of {\tt pickrell.eset} having gene symbol, chromosome start and end and entrez id. NOTES: genes are encoded using ensembl ids. Use the `Homo.sapiens` object to look up the chromosome start, chomosome end and entrez {\tt columns()}.

\item Download GEO data set GSE85426 by executing:

\begin{verbatim}
library(GEOquery)
gds <- getGEO("GSE40732")[[1]]
gds
\end{verbatim} 

Answer the following questions using this object
\begin{itemize}
\item how many samples are analyzed?
\item show the names of the variables (metadata) available in this study
\item how many asthmatics are in the database (variable {\tt characteristics\_ch1})?
\item how many genes are analyzed?
\item which is the name of the annotation?
\item create a boxplot showing the gene expression of `AB002294` (variable ID in the annoation - these are genebank numbers).
\end{itemize}

\end{enumerate}

\end{document}


%%%%%%%%%%%%%%%%%%%%%%%%%%%%%%%%%%%%%%%%%%%%%%%%%%%%%%%%%%%%%%%%%%%%%%%%%%%%%%%%%%%%%%%%%%%%%%%%%%%%%%%%%%%%%%%%%%%
